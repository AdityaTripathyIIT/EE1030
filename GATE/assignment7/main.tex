\let\negmedspace\undefined
\let\negthickspace\undefined
\documentclass[journal,12pt,onecolumn]{IEEEtran}
\usepackage{cite}
\usepackage{amsmath,amssymb,amsfonts,amsthm}
\usepackage{amsmath}
\usepackage{algorithmic}
\usepackage{graphicx}
\usepackage{textcomp}
\usepackage{xcolor}
\usepackage{txfonts}
\usepackage{listings}
\usepackage{multicol}
\usepackage{enumitem}
\usepackage{mathtools}
\usepackage{gensymb}
\usepackage{comment}
\usepackage[breaklinks=true]{hyperref}
\usepackage{tkz-euclide} 
\usepackage{listings}
\usepackage{gvv}                                        
\usepackage[latin1]{inputenc}                                
\usepackage{color}                                            
\usepackage{array}                                            
\usepackage{longtable}                                       
\usepackage{circuitikz}
\usepackage{calc}                                             
\usepackage{multirow}                                         
\usepackage{hhline}                                           
\usepackage{censor}                                           
\usepackage{ifthen}                                           
\usepackage{lscape}
\usepackage{tabularx}
\usepackage{array}
\usepackage{float}


\newtheorem{theorem}{Theorem}[section]
\newtheorem{problem}{Problem}
\newtheorem{proposition}{Proposition}[section]
\newtheorem{lemma}{Lemma}[section]
\newtheorem{corollary}[theorem]{Corollary}
\newtheorem{example}{Example}[section]
\newtheorem{definition}[problem]{Definition}
\newcommand{\BEQA}{\begin{eqnarray}}
\newcommand{\EEQA}{\end{eqnarray}}
\newcommand{\define}{\stackrel{\triangle}{=}}
\theoremstyle{remark}
\newtheorem{rem}{Remark}

\begin{document}
\bibliographystyle{IEEEtran}
\vspace{3cm}

\title{2017-ME-27-39}
\author{EE24BTECH11001 -  ADITYA TRIPATHY}
\maketitle

\renewcommand{\thefigure}{\theenumi}
\renewcommand{\thetable}{\theenumi}

\begin{enumerate}
    \item 
        For the vector $\vec{V} = 2yz\vec{i} + 3xz\vec{j} + 4xy\vec{k}$, the value of 
        $\nabla \cdot \brak{\nabla \times \vec{V}}$ is 
        \hfill{\brak{2017-ME}}
\\
    \item A parametric curve defined by $x = \cos\brak{\frac{\pi u}{2}}, y = \sin\brak{\frac{\pi u}{2}}$ 
        in the range $0 \le u \le 1$ is rotated about the $x$-axis by 360 degrees. Area of the surface 
        generated is

        \hfill{\brak{2017-ME}}
        \begin{multicols}{4}
            \begin{enumerate}
                \item $\frac{\pi}{2}$
                    \columnbreak
                \item $\pi$
                    \columnbreak
                \item $2\pi$
                    \columnbreak
                \item $4\pi$
            \end{enumerate}
        \end{multicols}
    \item $P\brak{0, 3}, Q\brak{0.5, 4},$ and $R\brak{1, 5}$ are three points on the defined by $f\brak{x}$.
        Numerical integration is carried out using bothe trapezoidal rule and Simpson's rule within limits
        $x = 0$ and $x = 1$ for the curve. The difference between the two results will be
        \hfill{\brak{2017-ME}}
        \begin{multicols}{4}
            \begin{enumerate}
                \item 0
                    \columnbreak
                \item 0.25
                    \columnbreak
                \item 0.5
                    \columnbreak
                \item 1
            \end{enumerate}
        \end{multicols}
    \item The velocity profile inside the boundary layer flow for flow over a flat plate is given by 
        $\frac{u}{u_{\infty}}$ is the free stream velocity and $\delta$ is the local boundary layer
        thickness, If $\delta ^*$ is the local displacement thickness, the value of $\frac{\delta ^ *}{\delta}$ is
        \hfill{\brak{2017-ME}}
        \begin{multicols}{4}
            \begin{enumerate}
                \item $\frac{2}{\pi}$
                    \columnbreak
                \item 1 - $\frac{2}{\pi}$
                    \columnbreak
                \item 1 + $\frac{2}{\pi}$
                    \columnbreak
                \item 0
            \end{enumerate}
        \end{multicols}
    \item Consider steady flow of an incompressible fluid through two long and straight pipes of 
        diameters $d_1$ and $d_2$ arranged in series. Both pipes are of equal lenght and the flow is 
        turbulent in both pipes. The friction factor for turbulent flow thorught pipes is of the form 
        $f = K\brak{Re}^{-n}$, where $K$ and $n$ are known positive constants and $Re$ is the Reynolds
        number. Neglecting minor losses, the ration of the frictional pressure drop in pipe 1 to 
        that in pipe 2, $\frac{\Delta P_1}{\Delta P_2}$, is given by
        \hfill{\brak{2017-ME}}

        \begin{multicols}{4}
            \begin{enumerate}
                \item $\brak{\frac{d_2}{d_1}}^{\brak{5-n}}$
                    \columnbreak
                \item $\brak{\frac{d_2}{d_1}}^5$
                    \columnbreak
                \item $\brak{\frac{d_2}{d_1}}^{\brak{3-n}}$
                    \columnbreak
                \item $\brak{\frac{d_2}{d_1}}^{\brak{5+n}}$
            \end{enumerate}
        \end{multicols}

    \item For a steady flow, the velocity field is $\vec{V} = \brak{-x^2 + 3z}\vec{i} + \brak{2xy}\vec{j}$.
        The magnitude of the acceleration of a particle at $\brak{1, -1}$ is
        \hfill{\brak{2017-ME}}
        \begin{multicols}{4}
            \begin{enumerate}
                \item 2
                    \columnbreak
                \item 1
                    \columnbreak
                \item $2\sqrt{5}$
                    \columnbreak
                \item 0
            \end{enumerate}
        \end{multicols}

    \item One kg of an ideal gas $\brak{\textnormal{gas constant, R = 400J/kg K; specific
        heat at constant volume} c_v = 1000 J/kg K}$ at 1 bar, and 300K is contained in a sealed rigid 
        cylinder. During an adiabatic processs, 100$kJ$ of work is done on the system by a stirrer. 
        The increase in entropy of the system is 
        \hfill{\brak{2017-ME}}
   \\     
    \item The pressure ratio across a gas turbine for air,(specific heat at constant pressure,$C_p = 1040J/kg K$ and ratio of specific heats $\gamma = 1.4$) is
        10. If the inlet temperature to the turbine is 1200 K and the isoentropic efficiency is 0.9, the gas 
        temperature at turbine exit is 
        \hfill{\brak{2017-ME}}
   \\     
        
    \item Moist air is treated as an ideal gas mixture of water vapor and dry air 
        (molecular weight of air = 28.84 and molecular weight of water = 18). At a 
        location, the total pressure is 100$kPa$, the temperature is $30 ^{\degree}$ and the 
        relative humidity is $55\%$. Given that the saturation pressure of water at $30 ^{\degree}$
        is $42646 Pa$, the mass of water vapor per kg of dry air is
        \hfill{\brak{2017-ME}}
        

    \item Air contains $79\% N_2$ and $21 \% O_2$ on a molar basis. Methane $\brak{CH_4}$ is 
        burned with $50\%$ excess air than required stoichiometrically. Assuming complete combustion 
        of methane, the molar percentage of $N_2$ in the products is
        \hfill{\brak{2017-ME}}
        
    \item Two black surfaces, $AB$ and $BC$, of lengths $5m$ amd $6m$, respectively, are oriented as shown.
        Both surfaces extend infinitely into the third dimension. Given that view factor 
        $F_{12} = 0.5, T_1 = 800K, T_2 = 600K, T_{surrounding} = 300K$ and Stefan Boltzman constant
        $\sigma = 5.67 \times 10^{-8} W/\brak{m^2 K^4}$ the heat transfer from Surface 2 to the surrounding
        environment is $\brak{\text{in} kW}$
        \hfill{\brak{2017-ME}}
        \begin{center}
            \resizebox{0.5\textwidth}{!}{
                \begin{circuitikz}
\tikzstyle{every node}=[font=\LARGE]
\draw (3.5,14.5) to[L ] (5.5,14.5);
\draw (3.5,14.5) to[sinusoidal voltage source, sources/symbol/rotate=auto] (3.5,12.5);
\draw (5.5,14.5) to[D] (5.5,16.25);
\draw (5.5,10.5) to[D] (5.5,12.25);
\draw (7.25,10.5) to[D] (7.25,12.25);
\draw (7.25,14.5) to[D] (7.25,16.25);
\draw [ line width=0.5pt](3.5,12.5) to[crossing] (7.25,12.5);
\draw (5.5,14.5) to[short] (5.5,12.25);
\draw (5.5,10.5) to[short] (7.25,10.5);
\draw (7.25,12.25) to[short] (7.25,14.5);
\draw (5.5,16.25) to[short] (9.5,16.25);
\draw (7.25,10.5) to[short] (9.5,10.5);
\draw (9.5,15) to[american current source] (9.5,12.75);
\draw (9.5,15) to[short] (9.5,16.25);
\draw (9.5,12.75) to[short] (9.5,10.5);
\node [font=\LARGE] at (1.5,13.5) {$220V,50Hz\quad$};
\node [font=\LARGE] at (3.5,15.0) {$L_s=10mH\quad$};
\node [font=\LARGE] at (4.5,16) {$D_1$};
\node [font=\LARGE] at (8,11.5) {$D_2$};
\node [font=\LARGE] at (4.5,11.5) {$D_4$};
\node [font=\LARGE] at (11,14) {$\quad I_0=14A$};
\node [font=\LARGE] at (8,15.5) {$D_3$};
\end{circuitikz}

            } 
        \end{center} 
    \item Heat is generated uniformly in a long solid cylindrical rod $\brak{\text{diameter} = 10mm}$
        at the rate of $4 \times 10^7 W/m^3$. The thermal conductivity of the rod material is 25 $W/m K$. Under
        steady state conditions, the temperature difference between the centre and the surface of the rod is
        $\brak{\text{in} \quad^{\degree}C}$
        \hfill{\brak{2017-ME}}

    \item An initially stress free massless elastic beam of length $L$ and circular cross-section
        with diameter $d$ \brak{d << L} is held fixed between two walls as showm. The beam material
        has Young's modulus E and coefficient of thermal expansion $\alpha$. If the beam is slowly and
        uniformly heated, the temperatture rise required to cause the beam to buckle is proportional to
        \begin{center}
            \resizebox{0.5\textwidth}{!}{
                \begin{circuitikz}
\tikzstyle{every node}=[font=\LARGE]
\draw [short] (2.25,13.5) -- (8.75,13.5);
\draw [short] (2.25,12.75) -- (8.75,12.75);
\draw [short] (2.25,14.5) -- (2.25,12);
\draw [short] (8.75,14.5) -- (8.75,12);
\draw [short] (2.25,14.25) .. controls (2,14) and (2,14.25) .. (1.75,13.75);
\node [font=\LARGE] at (2.5,13.5) {};
\node [font=\LARGE] at (2.5,13.5) {};
\draw [short] (2.25,13.5) -- (1.75,13);
\draw [short] (2.25,12.75) -- (1.75,12.25);
\draw [short] (8.75,14.25) -- (9.5,14.75);
\draw [short] (8.75,13.5) -- (9.5,14);
\draw [short] (8.75,12.5) -- (9.5,13);
\draw [->, >=Stealth] (6.5,15) -- (6.5,13.5);
\draw [->, >=Stealth] (6.5,11.5) -- (6.5,12.75);
\node [font=\LARGE] at (3.5,11.5) {L};
\node [font=\LARGE] at (6,15.5) {d};
\end{circuitikz}

            } 
        \end{center}
        \hfill{\brak{2017-ME}}
        \begin{multicols}{4}
            \begin{enumerate}
                \item $d$
                    \columnbreak
                \item $d^2$
                    \columnbreak
                \item $d^3$
                    \columnbreak
                \item $d^4$
            \end{enumerate}
        \end{multicols}
\end{enumerate}
\end{document}
