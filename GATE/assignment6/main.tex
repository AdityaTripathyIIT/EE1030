\let\negmedspace\undefined
\let\negthickspace\undefined
\documentclass[journal,12pt,onecolumn]{IEEEtran}
\usepackage{cite}
\usepackage{amsmath,amssymb,amsfonts,amsthm}
\usepackage{amsmath}
\usepackage{algorithmic}
\usepackage{graphicx}
\usepackage{textcomp}
\usepackage{xcolor}
\usepackage{txfonts}
\usepackage{listings}
\usepackage{multicol}
\usepackage{enumitem}
\usepackage{mathtools}
\usepackage{gensymb}
\usepackage{comment}
\usepackage[breaklinks=true]{hyperref}
\usepackage{tkz-euclide} 
\usepackage{listings}
\usepackage{gvv}                                        
\usepackage[latin1]{inputenc}                                
\usepackage{color}                                            
\usepackage{array}                                            
\usepackage{longtable}                                       
\usepackage{circuitikz}
\usepackage{calc}                                             
\usepackage{multirow}                                         
\usepackage{hhline}                                           
\usepackage{censor}                                           
\usepackage{ifthen}                                           
\usepackage{lscape}
\usepackage{tabularx}
\usepackage{array}
\usepackage{float}
\censorruledepth=-.2ex
\censorruleheight=.1ex

\newtheorem{theorem}{Theorem}[section]
\newtheorem{problem}{Problem}
\newtheorem{proposition}{Proposition}[section]
\newtheorem{lemma}{Lemma}[section]
\newtheorem{corollary}[theorem]{Corollary}
\newtheorem{example}{Example}[section]
\newtheorem{definition}[problem]{Definition}
\newcommand{\BEQA}{\begin{eqnarray}}
\newcommand{\EEQA}{\end{eqnarray}}
\newcommand{\define}{\stackrel{\triangle}{=}}
\theoremstyle{remark}
\newtheorem{rem}{Remark}

\begin{document}
\bibliographystyle{IEEEtran}
\vspace{3cm}

\title{2017-EE-53-65}
\author{EE24BTECH11001 -  ADITYA TRIPATHY}
\maketitle

\renewcommand{\thefigure}{\theenumi}
\renewcommand{\thetable}{\theenumi}

\begin{enumerate}
    \item 
        The figre below shows an uncontrolled diode bridge rectifier supplied from $220V, 50Hz$ 
        1-phase $ac$ source. The load draws a constant current $I_0 = 14A$. The conduction angle
        of the diode $D_1$ in degrees $\brak{\textnormal{rounded off to two decimal places}}$ is 
        \hfill{\brak{2017-EE}}
        \begin{center}
            \resizebox{0.5\textwidth}{!}{
                \begin{circuitikz}
\tikzstyle{every node}=[font=\LARGE]
\draw (3.5,14.5) to[L ] (5.5,14.5);
\draw (3.5,14.5) to[sinusoidal voltage source, sources/symbol/rotate=auto] (3.5,12.5);
\draw (5.5,14.5) to[D] (5.5,16.25);
\draw (5.5,10.5) to[D] (5.5,12.25);
\draw (7.25,10.5) to[D] (7.25,12.25);
\draw (7.25,14.5) to[D] (7.25,16.25);
\draw [ line width=0.5pt](3.5,12.5) to[crossing] (7.25,12.5);
\draw (5.5,14.5) to[short] (5.5,12.25);
\draw (5.5,10.5) to[short] (7.25,10.5);
\draw (7.25,12.25) to[short] (7.25,14.5);
\draw (5.5,16.25) to[short] (9.5,16.25);
\draw (7.25,10.5) to[short] (9.5,10.5);
\draw (9.5,15) to[american current source] (9.5,12.75);
\draw (9.5,15) to[short] (9.5,16.25);
\draw (9.5,12.75) to[short] (9.5,10.5);
\node [font=\LARGE] at (1.5,13.5) {$220V,50Hz\quad$};
\node [font=\LARGE] at (3.5,15.0) {$L_s=10mH\quad$};
\node [font=\LARGE] at (4.5,16) {$D_1$};
\node [font=\LARGE] at (8,11.5) {$D_2$};
\node [font=\LARGE] at (4.5,11.5) {$D_4$};
\node [font=\LARGE] at (11,14) {$\quad I_0=14A$};
\node [font=\LARGE] at (8,15.5) {$D_3$};
\end{circuitikz}

            }
        \end{center}
    \item The positive, negative and zero sequence reactances of a wye-connected synchronous generator
        are $0.2 pu, 0.2 pu$ and $0.1 pu$, respectively. The generator is on open circuit with terminal 
        voltage of $1 pu$. The minimum value of the inductive reactance, in $pu$, required to be 
        connected between neutral and ground so that the fault current does not exceed $2.75 pu$ is a single
        line to ground fault occurs at athe terminals is \\
        $\brak{\textnormal{assume fault impedance to be zero}}$. $\brak{\textnormal{Give the answer up to one decimal place}}$

        \hfill{\brak{2017-EE}}

    \item The figure shows the single line diagram of a apower system with a double circuit 
        transmission line. The expression for electrical power is $1.5 \sin \delta$, where
        $\delta$ is the rotor angle. The system is operation at the stable equilibrium point
        with mechanical power equal to 1$pu$. If one of the transmission line circuits is removed
        , the maximum value of $\delta$, as the rotor swings is $P_{max} \sin \delta$, the value
        of $P_{max}$, in $pu$ is \\
        $\brak{\textnormal{Give the answer up to three decimal place}}$

        \hfill{\brak{2017-EE}}
\begin{center}
            \resizebox{0.5\textwidth}{!}{
                \begin{circuitikz}
\tikzstyle{every node}=[font=\LARGE]
\draw [short] (2.25,13.5) -- (8.75,13.5);
\draw [short] (2.25,12.75) -- (8.75,12.75);
\draw [short] (2.25,14.5) -- (2.25,12);
\draw [short] (8.75,14.5) -- (8.75,12);
\draw [short] (2.25,14.25) .. controls (2,14) and (2,14.25) .. (1.75,13.75);
\node [font=\LARGE] at (2.5,13.5) {};
\node [font=\LARGE] at (2.5,13.5) {};
\draw [short] (2.25,13.5) -- (1.75,13);
\draw [short] (2.25,12.75) -- (1.75,12.25);
\draw [short] (8.75,14.25) -- (9.5,14.75);
\draw [short] (8.75,13.5) -- (9.5,14);
\draw [short] (8.75,12.5) -- (9.5,13);
\draw [->, >=Stealth] (6.5,15) -- (6.5,13.5);
\draw [->, >=Stealth] (6.5,11.5) -- (6.5,12.75);
\node [font=\LARGE] at (3.5,11.5) {L};
\node [font=\LARGE] at (6,15.5) {d};
\end{circuitikz}

            }
        \end{center}
    \item After Rajedra Chola returned from his voyage to Indonesia, he \xblackout{AAAAAAAA} to visit
        the temple in Thanjavur.
        \hfill{\brak{2017-EE}}
        \begin{multicols}{4}
            \begin{enumerate}
                \item was wishing
                    \columnbreak
                \item is wishing
                    \columnbreak
                \item wished
                    \columnbreak
                \item had wished
            \end{enumerate}
        \end{multicols}
    \item Research in the workplace reveals that people work for many reasons \xblackout{AAAAAAAAAA}.		
        \hfill{\brak{2017-EE}}

        \begin{multicols}{4}
            \begin{enumerate}
                \item money beside
                    \columnbreak
                \item beside money
                    \columnbreak
                \item money besides
                    \columnbreak
                \item besides money
            \end{enumerate}
        \end{multicols}

    \item Rahul, Muralu, Srinivas and Arul are seated around a square table. Rahul is sitting
        to the left of Murali, Srinivas is sitting to the right of Arul. Which of the following
        pairs are seated opposite each other ?
        \hfill{\brak{2017-EE}}
        \begin{enumerate}
                \begin{multicols}{2}
                \item Rahul and Murali
                    \columnbreak
                \item Srininvas and Arul
                \end{multicols}
                \begin{multicols}{2}
                \item Srinivas and Murali
                    \columnbreak
                \item Srinivas and Arul
                \end{multicols}
        \end{enumerate}

    \item Find the smallest number $y$ such that $y \times 162$ is a perfect cube.
        \hfill{\brak{2017-EE}}
        \begin{multicols}{4}
            \begin{enumerate}
                \item  24
                    \columnbreak
                \item 27
                    \columnbreak
                \item 32
                    \columnbreak
                \item 36
            \end{enumerate}
        \end{multicols}

    \item The probability that a $k$-digit number does NOT contain the digits 0, 5 or 9 is
        \hfill{\brak{2017-EE}}
        \begin{multicols}{4}
            \begin{enumerate}
                \item $0.3^k$
                    \columnbreak
                \item $0.6^k$
                    \columnbreak
                \item $0.7^k$
                    \columnbreak
                \item $0.9^k$
            \end{enumerate}
        \end{multicols}

    \item "The hold of the nationalist imagination on our colonial past is such that anything 
        inadequately or improperly nationalist is just not history."
        \\Which of the following statements best reflects the authos's opinion?
        \hfill{\brak{2017-EE}}
        \begin{enumerate}
            \item  Nationalists are highly imaginative.
            \item  History is viewed through the filter of nationalism.
            \item  Our colonial past never happened.
            \item  Nationalism has to be both adequatley and properly imagined.
        \end{enumerate}

    \item Six people are seated around a circular table. There are at least two men and two women. There
        are at least three right-handed persons. Every woman has a left-handed person to their immediate right.
        None of the women are right handed. The number of women at the table is 
        \hfill{\brak{2017-EE}}
        \begin{multicols}{4}
            \begin{enumerate}
                \item  2
                    \columnbreak
                \item 3
                    \columnbreak
                \item 4
                    \columnbreak
                \item Cannot be determined
            \end{enumerate}
        \end{multicols}

    \item The expression $\frac{\brak{x+y-|x-y|}}{2}$ is equal to  
        \hfill{\brak{2017-EE}}
        \begin{enumerate}
                \begin{multicols}{2}
                \item The maximum of x and y
                    \columnbreak
                \item The minimum of x and y
                \end{multicols}
                \begin{multicols}{2}
                \item 1
                    \columnbreak
                \item None of the above
                \end{multicols}
        \end{enumerate}    
    \item Arun, Gulab, Neel and Shweta must choose one shirt each from a pile of four shirts
        coloured red, pink, blue and white respectively. Arun dislikes the colour red and Shweta dislikes
        the colour white. Gulab and Neel like all the colours. In how many different ways can they choose the
        shirts so that no one has a shirt with a colour he or she dislikes
        \hfill{\brak{2017-EE}}
        \begin{multicols}{4}
            \begin{enumerate}
                \item 21
                    \columnbreak
                \item 18
                    \columnbreak
                \item 16
                    \columnbreak
                \item 14
            \end{enumerate}
        \end{multicols}
    \item A contour line joins locations having the smae height above the mean sea level. The following
        is a contour plot of a geographical region. Contour lines are shown at 25m intervals in this plot.
        If in a flood, the water rises to 525m, which of the villages P, Q, R, S, T get submerged?
        \hfill{\brak{2017-EE}}
        \begin{multicols}{4}
            \begin{enumerate}
                \item P, Q
                    \columnbreak
                \item P, Q, T
                    \columnbreak
                \item R, S, T
                    \columnbreak
                \item Q, R, S
            \end{enumerate}
        \end{multicols}
\end{enumerate}
\end{document}
