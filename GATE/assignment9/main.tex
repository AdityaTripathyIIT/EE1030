\let\negmedspace\undefined
\let\negthickspace\undefined
\documentclass[journal,12pt,onecolumn]{IEEEtran}
\usepackage{cite}
\usepackage{amsmath,amssymb,amsfonts,amsthm}
\usepackage{amsmath}
\usepackage{algorithmic}
\usepackage{graphicx}
\usepackage{textcomp}
\usepackage{xcolor}
\usepackage{txfonts}
\usepackage{listings}
\usepackage{multicol}
\usepackage{enumitem}
\usepackage{mathtools}
\usepackage{gensymb}
\usepackage{comment}
\usepackage[breaklinks=true]{hyperref}
\usepackage{tkz-euclide} 
\usepackage{listings}
\usepackage{gvv}                                        
\usepackage[latin1]{inputenc}                                
\usepackage{color}                                            
\usepackage{array}                                            
\usepackage{longtable}                                       
\usepackage{circuitikz}
\usepackage{calc}                                             
\usepackage{multirow}                                         
\usepackage{hhline}                                           
\usepackage{ifthen}                                           
\usepackage{lscape}
\usepackage{tabularx}
\usepackage{array}
\usepackage{float}

\newtheorem{theorem}{Theorem}[section]
\newtheorem{problem}{Problem}
\newtheorem{proposition}{Proposition}[section]
\newtheorem{lemma}{Lemma}[section]
\newtheorem{corollary}[theorem]{Corollary}
\newtheorem{example}{Example}[section]
\newtheorem{definition}[problem]{Definition}
\newcommand{\BEQA}{\begin{eqnarray}}
\newcommand{\EEQA}{\end{eqnarray}}
\newcommand{\define}{\stackrel{\triangle}{=}}
\theoremstyle{remark}
\newtheorem{rem}{Remark}

\begin{document}
\bibliographystyle{IEEEtran}
\vspace{3cm}

\title{2021-ME-1-13}
\author{EE24BTECH11001 -  ADITYA TRIPATHY}
\maketitle

\renewcommand{\thefigure}{\theenumi}
\renewcommand{\thetable}{\theenumi}

\begin{enumerate}
    \item 
        If $y\brak{x}$ satisfies the differential equation $\brak{\sin x}\frac{dy}{dx} + y\cos x = 1$,
        subject to the domain y$\brak{\frac{\pi}{2}} = \frac{\pi}{2}$, then y$\brak{\frac{\pi}{2}}$
        is 
        \hfill{\brak{2021-ME}}
        \begin{multicols}{4}
            \begin{enumerate}
                \item 0
                    \columnbreak
                \item $\frac{\pi}{6}$
                    \columnbreak
                \item $\frac{\pi}{3}$
                    \columnbreak
                \item $\frac{\pi}{2}$
            \end{enumerate}
        \end{multicols}
    \item The value of $\lim_{x \rightarrow 0} \frac{1 - \cos x}{x^2}$ is
        \hfill{\brak{2021-ME}}
        \begin{multicols}{4}
            \begin{enumerate}
                \item $\frac{1}{4}$
                    \columnbreak
                \item $\frac{1}{3}$
                    \columnbreak
                \item $\frac{1}{2}$
                    \columnbreak
                \item $1$
            \end{enumerate}
        \end{multicols}

    \item The Dirac-Delta function $\delta\brak{t -t_0}$ for $t, t_0, \in \mathbb{R}$, has the following
        property 
        \begin{align}
            \int_a^b \phi\brak{t}\delta{t - t_0} \, dt = \begin{cases}
                \phi\brak{t_0} & a < t_0 < b\\
                0 & otherwise
            \end{cases} 
        \end{align}
        The laplace transform of the Dirac-Delta function $\delta\brak{t - a}$ for $a > 0$, 
        $\mathcal{L}\brak{\delta\brak{t - a}} = F\brak{s}$ is 
        \hfill{\brak{2021-ME}}
        \begin{multicols}{4}
            \begin{enumerate}
                \item $0$
                    \columnbreak
                \item $\infty$
                    \columnbreak
                \item $e^{sa}$
                    \columnbreak
                \item $e^{-sa}$
            \end{enumerate}
        \end{multicols}
    \item The ordinary differential equation $\frac{dy}{dx} = -\pi y$ subject to an initial condition
        $y\brak{0} = 1$ is solved numerically using the following scheme:
        \begin{align}
            \frac{y\brak{t_{n+1}} - y\brak{t_n}}{h} = -\pi y\brak{t_n} 
        \end{align}
        where $h$ is the time step, $t_n = nh$, and $n = 0, 1, 2 \cdots$ . This numerical scheme
        is stable for all values of $h$ in the interval
        \hfill{\brak{2021-ME}}
        \begin{multicols}{4}
            \begin{enumerate}
                \item $0 < h < \frac{2}{\pi}$
                    \columnbreak
                \item $0 < h < 1$
                    \columnbreak
                \item $0 < h < \frac{\pi}{2}$
                    \columnbreak
                \item for all $h > 0$
            \end{enumerate}
        \end{multicols}

\item Consider a binomial random variable $X$. If $X_1, X_2, \cdots , X_n$ are independent and 
    identically distributed samples from the distribution of $X$ with sum $Y = \sum_{i = 1}^{n}X_i$,
    then distribution of $Y$ as $n \rightarrow \infty$ can be approximated as
        \hfill{\brak{2021-ME}}
        \begin{multicols}{4}
            \begin{enumerate}
                \item Exponential
                    \columnbreak
                \item Bernoulli
                    \columnbreak
                \item Binomial
                    \columnbreak
                \item Normal
            \end{enumerate}
        \end{multicols}
    \item The loading and unloading response of a metal is shown in the figure. The elastic and
        plastic strains corresponding to $200 MPa$ stress, respectively, are
        \begin{center}
            \resizebox{0.5\textwidth}{!}{
                \begin{circuitikz}
\tikzstyle{every node}=[font=\LARGE]
\draw (3.5,14.5) to[L ] (5.5,14.5);
\draw (3.5,14.5) to[sinusoidal voltage source, sources/symbol/rotate=auto] (3.5,12.5);
\draw (5.5,14.5) to[D] (5.5,16.25);
\draw (5.5,10.5) to[D] (5.5,12.25);
\draw (7.25,10.5) to[D] (7.25,12.25);
\draw (7.25,14.5) to[D] (7.25,16.25);
\draw [ line width=0.5pt](3.5,12.5) to[crossing] (7.25,12.5);
\draw (5.5,14.5) to[short] (5.5,12.25);
\draw (5.5,10.5) to[short] (7.25,10.5);
\draw (7.25,12.25) to[short] (7.25,14.5);
\draw (5.5,16.25) to[short] (9.5,16.25);
\draw (7.25,10.5) to[short] (9.5,10.5);
\draw (9.5,15) to[american current source] (9.5,12.75);
\draw (9.5,15) to[short] (9.5,16.25);
\draw (9.5,12.75) to[short] (9.5,10.5);
\node [font=\LARGE] at (1.5,13.5) {$220V,50Hz\quad$};
\node [font=\LARGE] at (3.5,15.0) {$L_s=10mH\quad$};
\node [font=\LARGE] at (4.5,16) {$D_1$};
\node [font=\LARGE] at (8,11.5) {$D_2$};
\node [font=\LARGE] at (4.5,11.5) {$D_4$};
\node [font=\LARGE] at (11,14) {$\quad I_0=14A$};
\node [font=\LARGE] at (8,15.5) {$D_3$};
\end{circuitikz}

            } 
        \end{center}
        \hfill{\brak{2021-ME}}

        \begin{multicols}{4}
            \begin{enumerate}
                \item 0.01 and 0.01
                    \columnbreak
                \item 0.02 and 0.01
                    \columnbreak
                \item 0.01 and 0.02
                    \columnbreak
                \item 0.02 and 0.02
            \end{enumerate}
        \end{multicols}

    \item In a machining operation, if a cutting tool traces the workpiece such that the directrix is
        perpendicular to the plane of the generatrix as shown in figure, the surface generated is 
        \begin{center}
            \resizebox{0.4\textwidth}{!}{
                \begin{circuitikz}
\tikzstyle{every node}=[font=\LARGE]
\draw [short] (2.5,18.25) -- (2.5,18.25);
\draw [short] (1.5,18.75) -- (1.5,17.75);
\draw [short] (1.5,18.25) -- (6.5,18.25);
\draw [short] (6.5,18.25) -- (6.5,13);
\draw [short] (5.75,13) -- (7.25,13);
\draw [<->, >=Stealth] (7.75,18.5) -- (7.75,13);
\draw [short] (5.75,13) -- (5.5,12.75);
\draw [short] (6,13) -- (5.75,12.75);
\draw [short] (6.25,13) -- (6,12.75);
\draw [short] (6.5,13) -- (6.25,12.75);
\draw [short] (6.75,13) -- (6.5,12.75);
\draw [short] (7,13) -- (6.75,12.75);
\draw [short] (7.25,13) -- (7,12.75);
\draw [short] (1.5,17.75) -- (1.25,17.5);
\draw [short] (1.5,18) -- (1.25,17.75);
\draw [short] (1.5,18.25) -- (1.25,18);
\draw [short] (1.5,18.5) -- (1.25,18.25);
\draw [short] (1.5,18.75) -- (1.25,18.5);
\draw [->, >=Stealth] (6.5,19) -- (6.5,18.5);
\node [font=\LARGE] at (2,19) {Q};
\node [font=\LARGE] at (6.5,19.5) {P};
\node [font=\LARGE] at (8.5,15.75) {L};
\node [font=\LARGE] at (6,15.75) {EI};
\node [font=\LARGE] at (6,13.5) {S};
\draw [<->, >=Stealth] (1.75,12.25) -- (6.5,12.25);
\node [font=\LARGE] at (6,18.5) {R};
\node [font=\LARGE] at (3.75,11.75) {L};
\end{circuitikz}

            } 
        \end{center}
        \hfill{\brak{2021-ME}}
        \begin{multicols}{4}
            \begin{enumerate}
                \item plane
                    \columnbreak
                \item cylindrical
                    \columnbreak
                \item spherical
                    \columnbreak
                \item a surface of revolution
            \end{enumerate}
        \end{multicols}


    \item The correct sequence of machining operations to be performed to finish a large diameter
        through hole is
        \hfill{\brak{2021-ME}}
            \begin{enumerate}
        \begin{multicols}{2}
                \item drilling, boring, reaming
                    \columnbreak
                \item boring, drilling, reaming
        \end{multicols}
        \begin{multicols}{2}
                \item drilling, reaming, boring
                    \columnbreak
                \item boring, reaming, drilling
        \end{multicols}
            \end{enumerate}

    \item In modern CNC machine tools, the backlash has been eliminated by
        \hfill{\brak{2021-ME}}
            \begin{enumerate}
        \begin{multicols}{2}
                \item preloaded ballscrews
                    \columnbreak
                \item rack and pinion
        \end{multicols}
        \begin{multicols}{2}
                \item ratchet and pinion
                    \columnbreak
                \item slider crank mechanism
        \end{multicols}
            \end{enumerate}

    \item Consider the surface roughness profile as shown in the figure. The center line average 
        roughness ($R_a$, in $\mu$m)of the measured length (L) is
        \begin{center}
            \resizebox{0.5\textwidth}{!}{
                \begin{circuitikz}
\tikzstyle{every node}=[font=\LARGE]
\draw [short] (2.75,17.5) -- (2.75,16.5);
\draw [short] (2.75,17.5) -- (7.25,17.5);
\draw [short] (7.25,17.5) -- (7.25,16.5);
\draw [short] (2.75,16.5) -- (4.5,16.5);
\draw [short] (7.25,16.5) -- (5.75,16.5);
\draw [short] (5.75,16.5) -- (5.75,13.25);
\draw [short] (4.5,16.5) -- (4.5,11.75);
\draw [short] (5.75,13.25) -- (5.75,11.75);
\draw [short] (5.75,11.75) -- (4.5,11.75);
\draw [<->, >=Stealth] (7,16.25) -- (7,11.75);
\draw [<->, >=Stealth] (8,17.5) -- (8,16.5);
\draw [<->, >=Stealth] (2.75,18) -- (7.75,18);
\draw [<->, >=Stealth] (2.25,17.5) -- (2.25,14.5);
\draw [dashed] (3,14.5) -- (7,14.5);
\node [font=\LARGE] at (1.25,16.25) {y};
\node [font=\LARGE] at (5,18.5) {60};
\node [font=\LARGE] at (3.5,15) {N.A.};
\node [font=\LARGE] at (7.75,14.75) {60};
\node [font=\LARGE] at (9,17) {5};
\draw [<->, >=Stealth] (4.5,11.25) -- (5.75,11.25);
\node [font=\LARGE] at (5,10.5) {5};
\end{circuitikz}

            } 
        \end{center}
        \hfill{\brak{2021-ME}}
        \begin{multicols}{4}
            \begin{enumerate}
                \item 0
                    \columnbreak
                \item 1
                    \columnbreak
                \item 2
                    \columnbreak
                \item 4
            \end{enumerate}
        \end{multicols}

    \item In which of the following pairs of cycles, both cycles have at least one isothermal
        process?
        \hfill{\brak{2021-ME}}
            \begin{enumerate}
        \begin{multicols}{2}
                \item  Diesel cycle and Otto cycle
                    \columnbreak
                \item Carnot cycle and Stirling cycle
        \end{multicols}
        \begin{multicols}{2}
                \item Brayton cycle and Rankine cycle
                    \columnbreak
                \item Bell-Coleman cycle and Vapour compression refrigeration cycle
        \end{multicols}
            \end{enumerate}

    \item Supeheated steam at 1500$kPa$, has a specific volume of $2.75 m^3/kmol$ and compressibility
        factor $\brak{Z}$ of 0.95 . The temperature of steam is (in $\quad^{\degree}C$)\\
        (round off to the nearest integer).
        \hfill{\brak{2021-ME}}
        \begin{multicols}{4}
            \begin{enumerate}
                \item 522
                    \columnbreak
                \item 471
                    \columnbreak
                \item 249
                    \columnbreak
                \item 198
            \end{enumerate}
        \end{multicols}

    \item A hot steel spherical ball is suddenly dipped into a low temperature oil bath. Which of 
        the following dimensionless parameters are required to determine instantaneous center temperature
        of the ball using a Heisler chart?
        \hfill{\brak{2021-ME}}
            \begin{enumerate}
        \begin{multicols}{2}
                \item Biot number and Fourier number
                    \columnbreak
                \item Reynolds Number and Prandtl number
        \end{multicols}
        \begin{multicols}{2}
                \item Biot number and Froude number
                    \columnbreak
                \item Nusselt number and Grashoff number
        \end{multicols}
            \end{enumerate}
    
\end{enumerate}
\end{document}
