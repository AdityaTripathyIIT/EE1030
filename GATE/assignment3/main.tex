\let\negmedspace\undefined
\let\negthickspace\undefined
\documentclass[journal,12pt,onecolumn]{IEEEtran}
\usepackage{cite}
\usepackage{amsmath,amssymb,amsfonts,amsthm}
\usepackage{amsmath}
\usepackage{algorithmic}
\usepackage{graphicx}
\usepackage{textcomp}
\usepackage{xcolor}
\usepackage{txfonts}
\usepackage{listings}
\usepackage{multicol}
\usepackage{enumitem}
\usepackage{mathtools}
\usepackage{gensymb}
\usepackage{comment}
\usepackage[breaklinks=true]{hyperref}
\usepackage{tkz-euclide} 
\usepackage{listings}
\usepackage{gvv}                                        
\usepackage[latin1]{inputenc}                                
\usepackage{color}                                            
\usepackage{array}                                            
\usepackage{longtable}                                       
\usepackage{calc}                                             
\usepackage{multirow}                                         
\usepackage{hhline}                                           
\usepackage{ifthen}                                           
\usepackage{lscape}
\usepackage{tabularx}
\usepackage{array}
\usepackage{float}


\newtheorem{theorem}{Theorem}[section]
\newtheorem{problem}{Problem}
\newtheorem{proposition}{Proposition}[section]
\newtheorem{lemma}{Lemma}[section]
\newtheorem{corollary}[theorem]{Corollary}
\newtheorem{example}{Example}[section]
\newtheorem{definition}[problem]{Definition}
\newcommand{\BEQA}{\begin{eqnarray}}
\newcommand{\EEQA}{\end{eqnarray}}
\newcommand{\define}{\stackrel{\triangle}{=}}
\theoremstyle{remark}
\newtheorem{rem}{Remark}

\begin{document}
\bibliographystyle{IEEEtran}
\vspace{3cm}

\title{2013-ME-14-26}
\author{EE24BTECH11001 -  ADITYA TRIPATHY}
\maketitle

\renewcommand{\thefigure}{\theenumi}
\renewcommand{\thetable}{\theenumi}

\begin{enumerate}
    \item[14.] 
        In simple exponential smoothing forecasting, to give higher weightage to recent demand
        information, the smoothing coefficient must be close to 
        \hfill{\brak{2013-ME}}
        \begin{multicols}{4}
            \begin{enumerate}
                \item -1
                    \columnbreak
                \item zero
                    \columnbreak
                \item 0.5
                    \columnbreak
                \item 6
            \end{enumerate}
        \end{multicols}

    \item[15.] A steel bar 200$mm$ in diameter is turned at a feed of $0.25 mm/rev$ with a depth of
        cut of 4$mm$. The rotational speed of the workpiece is 160$rpm$. The material removal rate
        in $mm^3 /s$ is
        \hfill{\brak{2013-ME}}
        \begin{multicols}{4}
            \begin{enumerate}
                \item 160 \columnbreak
                \item 167.6 \columnbreak
                \item 1600 \columnbreak
                \item 1.0
            \end{enumerate}
        \end{multicols}

    \item[16.] A cube shaped casting solidifies in 5 $min$. The solidification time in $min$
        for a cube of the same material, which is 8 times heavier than the original casting,
        will be
        \hfill{\brak{2013-ME}}
        \begin{multicols}{4}
        \begin{enumerate}
            \item 10 \columnbreak
            \item 20 \columnbreak
            \item 24 \columnbreak
            \item 40
        \end{enumerate}
\end{multicols}

    \item[17.] For a ductile material, toughness is a measure of
        \hfill{\brak{2013-ME}}
        \begin{enumerate}
                \begin{multicols}{2}
                \item resistance to scratching \columnbreak
                \item ability to absorb energy up to fracture
                \end{multicols}
                \begin{multicols}{2}
                \item ability to absorb energy till elastic limit \columnbreak
                \item resistance to indentation
                \end{multicols}
        \end{enumerate}

        item[18.] In order to have maximum power from Pelton turbine, the bucket speed must be
        \hfill{\brak{2013-ME}}
        \begin{enumerate}
                \begin{multicols}{2}
                \item equal to jet speed \columnbreak
                \item equal to half the jet speed
                \end{multicols}
                \begin{multicols}{2}
                \item equal to twice the jet speed \columnbreak
                \item independent of the jet speed
                \end{multicols}
        \end{enumerate}


    \item[19.]  Consider one-dimensional steady state heat conduction along x-axis $0 \le x \le L$
        , through a plane wall with the boundary surfaces $x = 0 \textnormal{and} x = L$ maintained
        at temperatures of $0^{\degree}C$ and $100^{\degree}C$. Heat is generated uniformly throughout
        the wall. Choos the \textbf{CORRECT} statement.

        \hfill{\brak{2013-ME}}
        \begin{enumerate}
            \item The direction of heat transfer will be from the surface at $100^{\degree}C$ to the surface at $0^{\degree}C$.
            \item The maximum temperature inside the wall must be greater than $100^{\degree}C$.
            \item The temperature distribution is linear within the wall.
            \item The temperature distribution is symmetric about the mid-plane of the wall.
        \end{enumerate}


    \item[20.] A cylinder contains 5$m^3$ of an ideal gas at a pressure of 1 $bar$. This gas is
        compressed in a reversible isothermal process till its pressure increases to 5 $bar$. The
        work in $kJ$ required for this process is
        \hfill{\brak{2013-ME}}
        \begin{multicols}{4}
            \begin{enumerate}
                \item 804.7 \columnbreak
                \item 953.2 \columnbreak
                \item 981.7 \columnbreak
                \item 1012.2
            \end{enumerate}
        \end{multicols}

    \item[21.] A long thin walled cylindrical shell, closed at both the ends, i ssubjected to an 
        internal pressures. The ratio of the hoop stress $\brak{\textnormal{circumferential stress}}$
        to the longitudinal stress developed in the shell is
        \hfill{\brak{2013-ME}}
        \begin{multicols}{4}
            \begin{enumerate}
                \item 0.5 \columnbreak
                \item 1.0 \columnbreak
                \item 2.0 \columnbreak
                \item 4.0
            \end{enumerate}
        \end{multicols}

    \item[22.] If two nodes are observed at a frequency of  1800$rpm$ during whirling of a simply
        supported long slender rotating shaft, the first critical speed of the shaft in $rpm$ is
        \hfill{\brak{2013-ME}}
        \begin{multicols}{4}
            \begin{enumerate}
                \item 200\columnbreak
                \item 450 \columnbreak
                \item 600 \columnbreak
                \item 900
            \end{enumerate}
        \end{multicols}

    \item[23.] A planar closed kinematic chain is formed with rigid links $PQ = 2.0m, QR = 3.0m, RS = 2.5m$ and
        $SP = 2.7m$ with all revolute joints. The link to be fixed to obtain a double rocker $\brak{\textnormal{rocker - rocker}}$
        mechanism is
        \hfill{\brak{2013-ME}}
        \begin{enumerate}
                \begin{multicols}{4}
                \item $PQ$ 
                \item $QR$
                \item $RS$
                \item $SP$ 
                \end{multicols}
        \end{enumerate}
    \item[24.] Let $X$ be a normal random bvariable with mean 1 and variance 4. The 
        probability $\Pr\brak{X < 0}$ is 
        \hfill{\brak{2013-ME}}
        \begin{enumerate}
                \begin{multicols}{2}
                \item 0.5 \columnbreak
                \item greater than zero and less than
                \end{multicols}
                \begin{multicols}{2}
                \item greater than 0.5 and less than 1.0\columnbreak
                \item 1.0
                \end{multicols}
        \end{enumerate}

    \item[25.] Choose The \textbf{CORRECT} set of functions, which are linearly dependent. 
        \hfill{\brak{2013-ME}}

        \begin{enumerate}
                \begin{multicols}{2}
                \item $\sin x, \sin ^2 x \textnormal{and} \cos ^2 x$ \columnbreak
                \item $\cos x, \sin x \textnormal{and} \tan x$ 
                \end{multicols}
                \begin{multicols}{2}
                \item $\cos 2x, \sin ^2  x \textnormal{and} \cos ^2 x$ \columnbreak
                \item $\cos 2x, \sin x \textnormal{and} \cos x$ 

                \end{multicols}
        \end{enumerate}
        \textbf{Q.26 to Q.55 carry two marks each.}\\
    \item[26.] The following surface integral is to be evaluated over a sphere for the 
        given steady velocity vector field $F = x\vec{i} + y\vec{j} + z\vec{k}$ defined
        with respect to a Cartesian corrdinate system having $\vec{i}, \vec{j} \textnormal{and} \vec{k}$
        as unit base vectors.
        \begin{align}
            \iint_S \frac{1}{4} \brak{\vec{F} \cdot \vec{n}} \, dA 
        \end{align}
        where $S$ is the sphere, $x^2 + y^2 + z^2 = 1$ and $\vec{n}$ is the outward
        unit normal vector to the sphere. The value of the surface integral is
        \hfill{\brak{2013-ME}}
        \begin{multicols}{4}
            \begin{enumerate}
                \item $\pi$ 
                    \columnbreak
                \item $2\pi$
                    \columnbreak
                \item $\frac{3\pi}{4}$
                    \columnbreak
                \item $4\pi$
            \end{enumerate}
        \end{multicols}
\end{enumerate}
\end{document}
