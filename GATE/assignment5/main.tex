\let\negmedspace\undefined
\let\negthickspace\undefined
\documentclass[journal,12pt,onecolumn]{IEEEtran}
\usepackage{cite}
\usepackage{amsmath,amssymb,amsfonts,amsthm}
\usepackage{amsmath}
\usepackage{algorithmic}
\usepackage{graphicx}
\usepackage{textcomp}
\usepackage{xcolor}
\usepackage{txfonts}
\usepackage{listings}
\usepackage{multicol}
\usepackage{enumitem}
\usepackage{mathtools}
\usepackage{gensymb}
\usepackage{circuitikz}
\usepackage{comment}
\usepackage[breaklinks=true]{hyperref}
\usepackage{tkz-euclide} 
\usepackage{listings}
\usepackage{gvv}                                        
\usepackage[latin1]{inputenc}                                
\usepackage{color}                                            
\usepackage{array}                                            
\usepackage{longtable}                                       
\usepackage{calc}                                             
\usepackage{multirow}                                         
\usepackage{hhline}                                           
\usepackage{ifthen}                                           
\usepackage{lscape}
\usepackage{tabularx}
\usepackage{array}
\usepackage{float}


\newtheorem{theorem}{Theorem}[section]
\newtheorem{problem}{Problem}
\newtheorem{proposition}{Proposition}[section]
\newtheorem{lemma}{Lemma}[section]
\newtheorem{corollary}[theorem]{Corollary}
\newtheorem{example}{Example}[section]
\newtheorem{definition}[problem]{Definition}
\newcommand{\BEQA}{\begin{eqnarray}}
\newcommand{\EEQA}{\end{eqnarray}}
\newcommand{\define}{\stackrel{\triangle}{=}}
\theoremstyle{remark}
\newtheorem{rem}{Remark}

\begin{document}
\bibliographystyle{IEEEtran}
\vspace{3cm}

\title{2014-XE-14-26}
\author{EE24BTECH11001 -  ADITYA TRIPATHY}
\maketitle

\renewcommand{\thefigure}{\theenumi}
\renewcommand{\thetable}{\theenumi}

\begin{enumerate}
    \item 
        Polymerized isotactic polybutadiene has a molecular weight of $3 \times 10^5 g/mol$. The degree of
        polymerization is
        \hfill{\brak{2014-XE}}
        \\
    \item  A bar of $Ti$ with Young's modulus of 110 $GPa$ and yield strength of $880 MPa$ is tested in tension. It
        is noticed that the alloy does not exhibit any strain hardening and fails at a total strain of 0.108. The
        mechanical energy that is necessary to break the material in $MJ/m^3$ is		
        \hfill{\brak{2014-XE}}
        \\
    \item A copper cup weighing 140 g contains 80 g of water at 4 $^{\degree}C$. Specific heats of water and copper are
        4.18 and 0.385 J/g $^{\degree}C$, respectively. If 100 g of water that is at 90 $^{\degree}C$ is added to the cup, the final
        temperature of water in $^{\degree}C$ is
        \hfill{\brak{2014-XE}}
        \\

    \item Match the reaction in $\textbf{Column I}$ with its name in $\textbf{Column II}$.\\
        L - liquid $\alpha, \beta, \gamma -$ different solid solution phases

        \begin{multicols}{2}
            \begin{enumerate}
                \item[] \textbf{Column I} \columnbreak
                \item[] \textbf{Column II}
            \end{enumerate}
        \end{multicols}
        \begin{multicols}{2}
            \begin{enumerate}
                \item[P.] L $\xrightarrow{cooling}$ $\quad \alpha + \beta$ \columnbreak
                \item[1.] peritectic
            \end{enumerate}

        \end{multicols}
        \begin{multicols}{2}
            \begin{enumerate}
                \item[Q.] L $+ \beta \xrightarrow{cooling}$ $\quad \gamma$ \columnbreak
                \item[2.] eutectic
            \end{enumerate}

        \end{multicols}
        \begin{multicols}{2}
            \begin{enumerate}
                \item[R.]  $\alpha \xrightarrow{cooling}$ $\quad \beta + \gamma$ \columnbreak
                \item[3.] monotectic
            \end{enumerate}

        \end{multicols}
        \begin{multicols}{2}
            \begin{enumerate}
                \item[]\columnbreak
                \item[1.] eutectoid
            \end{enumerate}

        \end{multicols}
        \hfill{\brak{2014-XE}}
        \begin{enumerate}
                \begin{multicols}{2}
                \item P-1, Q-4, R-3 \columnbreak 
                \item P-2, Q-1, R-4 
                \end{multicols} 
                \begin{multicols}{2}
                \item P-2, Q-3, R-1 \columnbreak 
                \item P-4, Q-2, R-3 
                \end{multicols}
        \end{enumerate}
    \item The Young's modulus of a unidirectional $SiC$ fiber reinforced $Ti$ matrix is 185 $GPa$.
        If the Young's modulii of $Ti$ and $SiC$ are 110 and 360 $GPa$ respectively, the volume fraction
        of fibers in the composite is 

        \hfill{\brak{2014-XE}}



    \item Match the composite in \textbf{Column I} with the most suitable application in 
        \textbf{Column II}
        \begin{multicols}{2}
            \begin{enumerate}
                \item[] \textbf{Column I} \columnbreak
                \item[] \textbf{Column II}
            \end{enumerate}
        \end{multicols}

        \begin{multicols}{2}
            \begin{enumerate}
                \item[P.] Glass fibre reinforced plastic \columnbreak
                \item[1.] Missile cone heads
            \end{enumerate}

        \end{multicols}
        \begin{multicols}{2}
            \begin{enumerate}
                \item[Q.] $SiC$ particle reiforced $Al$ alloy \columnbreak
                \item[2.] Commercial automobile chasis
            \end{enumerate}

        \end{multicols}
        \begin{multicols}{2}
            \begin{enumerate}
                \item[R.]  Carbon-carbon composite \columnbreak
                \item[3.] Airplane wheel tyres
            \end{enumerate}

        \end{multicols}
        \begin{multicols}{2}
            \begin{enumerate}
                \item[S.] Metal fibre reinforced rubber\columnbreak
                \item[4.]  Car piston rings 
            \end{enumerate}

        \end{multicols}
        \begin{multicols}{2}
            \begin{enumerate}
                \item[]\columnbreak
                \item[5.]  High performance skate boards
            \end{enumerate}

        \end{multicols}

        \hfill{\brak{2014-XE}}
        \begin{enumerate}
                \begin{multicols}{2}
                \item P-4, Q-5, R-1, S-2 \columnbreak 
                \item P-3, Q-5, R-2, S-4 
                \end{multicols} 
                \begin{multicols}{2}
                \item P-5, Q-4, R-1, S-3 \columnbreak 
                \item P-4, Q-2, R-3, S-1
                \end{multicols}
        \end{enumerate}

    \item Which among the following rules need to be satisfied for obtaining an isomorphous phase diagram
        in a binary alloy system? 
        \begin{enumerate}
            \item[P.] The atomic size difference should be less than $15\%$ 
            \item[Q.] Both the end components should have the same crystal structure
            \item[R.] The valency of the end components should be the same
            \item[S.] The end components should have dissimilar electronegativities
        \end{enumerate}
        \hfill{\brak{2014-XE}}
        \begin{multicols}{4}
            \begin{enumerate}
                \item  P, Q, R \columnbreak
                \item  Q, R, S \columnbreak
                \item  R, S, P  \columnbreak
                \item  S, P, Q
            \end{enumerate}
        \end{multicols}


    \item The energy in $eV$ and the wavelength in $\mu$m, respectively, of the photon emitted when an electron
        in a hydrogen atom falls from $n = 4$ to $n = 2$ state is
        \hfill{\brak{2014-XE}}
        \begin{multicols}{4}
            \begin{enumerate}
                \item  3.0, 0.413 \columnbreak
                \item  2.55, 0.365 \columnbreak
                \item  2.75, 0.451  \columnbreak
                \item  2.55, 0.487
            \end{enumerate}
        \end{multicols}

    \item  The weight in $kg$ of gallium $\brak{Ga}$ to be mixed with arsenic $\brak{As}$ for obtaining 1.0 kg of gallium
        arsenide $\brak{GaAs}$ is
        %\raggedright{M_{Ga} = 69.72 g/mol\; M_{As} = 74.92 g/mol}
        \hfill{\brak{2014-XE}}

    \item Match the material in \textbf{Column I} with the property in \textbf{Column II}
        \begin{multicols}{2}
            \begin{enumerate}
                \item[] \textbf{Column I} \columnbreak
                \item[] \textbf{Column II}
            \end{enumerate}
        \end{multicols}


        \begin{multicols}{2}
            \begin{enumerate}
                \item[P.]  $Pb\brak{Zr, Ti}O_3$\columnbreak
                \item[1.] Shape memory alloy
            \end{enumerate}

        \end{multicols}
        \begin{multicols}{2}
            \begin{enumerate}
                \item[Q.] $Ni_{50}Ti_{50} $\columnbreak
                \item[2.] Piezoelectric ceramic
            \end{enumerate}

        \end{multicols}
        \begin{multicols}{2}
            \begin{enumerate}
                \item[R.]  $GaAs$ \columnbreak
                \item[3.] High temperature superconductor
            \end{enumerate}

        \end{multicols}
        \begin{multicols}{2}
            \begin{enumerate}
                \item[S.] $YBa_2Cu_3O_7$\columnbreak
                \item[4.]  Optoelectronic semiconductor 
            \end{enumerate}

        \end{multicols}


        \hfill{\brak{2014-XE}}
        \begin{enumerate}
                \begin{multicols}{2}
                \item P-4, Q-5, R-1, S-2 \columnbreak 
                \item P-3, Q-5, R-2, S-4 
                \end{multicols} 
                \begin{multicols}{2}
                \item P-5, Q-4, R-1, S-3 \columnbreak 
                \item P-4, Q-2, R-3, S-1
                \end{multicols}
        \end{enumerate}

        \hfill{\brak{2014-XE}}

    \item Relevant portion of a binary phase diagram of elements A and B is shown below. The mass fraction
        of liquid phase at $1000 ^ {\degree}C$ for an alloy with 15 wt.$\% B$ is
        \hfill{\brak{2014-XE}}
        \begin{center}
            \resizebox{0.5\textwidth}{!}{%
                \begin{circuitikz}
\tikzstyle{every node}=[font=\LARGE]
\draw (3.5,14.5) to[L ] (5.5,14.5);
\draw (3.5,14.5) to[sinusoidal voltage source, sources/symbol/rotate=auto] (3.5,12.5);
\draw (5.5,14.5) to[D] (5.5,16.25);
\draw (5.5,10.5) to[D] (5.5,12.25);
\draw (7.25,10.5) to[D] (7.25,12.25);
\draw (7.25,14.5) to[D] (7.25,16.25);
\draw [ line width=0.5pt](3.5,12.5) to[crossing] (7.25,12.5);
\draw (5.5,14.5) to[short] (5.5,12.25);
\draw (5.5,10.5) to[short] (7.25,10.5);
\draw (7.25,12.25) to[short] (7.25,14.5);
\draw (5.5,16.25) to[short] (9.5,16.25);
\draw (7.25,10.5) to[short] (9.5,10.5);
\draw (9.5,15) to[american current source] (9.5,12.75);
\draw (9.5,15) to[short] (9.5,16.25);
\draw (9.5,12.75) to[short] (9.5,10.5);
\node [font=\LARGE] at (1.5,13.5) {$220V,50Hz\quad$};
\node [font=\LARGE] at (3.5,15.0) {$L_s=10mH\quad$};
\node [font=\LARGE] at (4.5,16) {$D_1$};
\node [font=\LARGE] at (8,11.5) {$D_2$};
\node [font=\LARGE] at (4.5,11.5) {$D_4$};
\node [font=\LARGE] at (11,14) {$\quad I_0=14A$};
\node [font=\LARGE] at (8,15.5) {$D_3$};
\end{circuitikz}
           
                }%
        \end{center}
    \item The expected diffraction angle $\brak{\textnormal{in degrees}}$for the first order reflection from the $\brak{113}$ set of planes
        for face centered cubic Pt $\brak{\textnormal{lattice parameter = 0.392 nm}}$ using monochromatic radiation of
        wavelength 0.1542 nm is
        \hfill{\brak{2014-XE}}
    \item The diffusion coefficients of i$Mg$ in $Al$ at 500 and $550^{\degree}C$ are $1.9\times 10^{-13}$
        and $5.8\times 10^{-13} m^2/s$ respectively. The activation energy for diffusion of $Mg$ in $Al$ in $kJ/mol$ is
        \hfill{\brak{2014-XE}}
\end{enumerate}
\end{document}
